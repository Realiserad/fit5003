\documentclass[11pt,twoside]{article}
\usepackage[T1]{fontenc}
\usepackage[latin1]{inputenc}
\usepackage[english]{babel}
\usepackage{amsmath}
\usepackage{amscd}
\usepackage{amssymb}
\usepackage{multirow}
\usepackage{tabularx}
\usepackage{url}
\usepackage{fancyhdr}
\usepackage{lastpage}
\usepackage[a4paper,margin=2.5cm,hmarginratio=1:1]{geometry}
\usepackage[framemethod=TikZ]{mdframed}
\usepackage{lipsum}
\mdfdefinestyle{MyFrame}{%
    linecolor=blue,
    outerlinewidth=2pt,
    roundcorner=20pt,
    innertopmargin=\baselineskip,
    innerbottommargin=\baselineskip,
    innerrightmargin=20pt,
    innerleftmargin=20pt,
    backgroundcolor=gray!50!white}

%%%%%%%%%%%%%%%%%%%%%%%%%%%%%%%%%%%%%%%%%%%%%%%%%%%%%%%%%%%%%%%%%%%%%%%%%%%
%%%%%%%%%%%%%% ENTER YOUR PERSONAL INFORMATION HERE %%%%%%%%%%%%%%%%%%%%%%%
%%%%%%%%%%%%%%%%%%%%%%%%%%%%%%%%%%%%%%%%%%%%%%%%%%%%%%%%%%%%%%%%%%%%%%%%%%%


% Your name, personal number, and email.
\newcommand{\firstname}{Bastian Fredriksson}
\newcommand{\email}{bfre12@student.monash.edu}


%%%%%%%%%%%%%%%%%%%%%%%%%%%%%%%%%%%%%%%%%%%%%%%%%%%%%%%%%%%%%%%%%%%%%%%%%%%
%%%%%%%%%%%%% DO NOT TOUCH ANYTHING BELOW THIS LINE %%%%%%%%%%%%%%%%%%%%%%%
%%%%%%%%%%%%%%%%%%%%%%%%%%%%%%%%%%%%%%%%%%%%%%%%%%%%%%%%%%%%%%%%%%%%%%%%%%%

\newcounter{problem}
\renewcommand\theproblem{\arabic{problem}}
\newenvironment{problem}{%
  \bigbreak
  \refstepcounter{problem}\noindent
  \llap{\textbf{\theproblem}\quad}\ignorespaces
}{%
  \par\if@nadaten@solutions\relax\else\filbreak\fi
}
\newenvironment{problem*}{%
  \bigbreak
  \refstepcounter{problem}\noindent
  \llap{\textbf{\theproblem}\quad}\ignorespaces
}{%
  \par
}
\newcounter{subproblem}[problem]
\renewcommand{\thesubproblem}{\arabic{problem}\alph{subproblem}}
\newenvironment{subproblem}{%
  \refstepcounter{subproblem}%
  \list{}{}%
  \item\leavevmode
  \llap{\hbox to \leftmargini{\textbf{\thesubproblem}\hfil}}%
  \ignorespaces
}{%
  \endlist\if@nadaten@solutions\relax\else\filbreak\fi
}
\newenvironment{subproblem*}{%
  \refstepcounter{subproblem}%
  \list{}{}%
  \item\leavevmode
  \llap{\hbox to \leftmargini{\textbf{\thesubproblem}\hfil}}%
  \ignorespaces
}{%
  \endlist
}

\newcommand{\homeworknr}{1}
\newcommand{\studentname}{\firstname~\lastnames}
\newcommand{\homework}{Assignment~\homeworknr}
\newcommand{\coursenumber}{FIT5003}
\newcommand{\coursename}{\coursenumber~Software security}
\newcommand{\coursenick}{fit5003}

\lhead[\studentname]{\coursename}
\chead{}
\rhead[\coursename]{\studentname}
\lfoot[\thepage~(\pageref{LastPage})]{}
\cfoot{}
\rfoot[]{\thepage~(\pageref{LastPage})}

\fancypagestyle{firststyle}
{
   \fancyhf{}
   \fancyfoot[R]{\thepage~(\pageref{LastPage})}
}

\renewcommand{\headrulewidth}{0pt}


%%%%%%%%%%%%%%%%%%%%%%%%%%%%%%%%%%%%%%%%%%%%%%%%%%%%%%%%%%%%%%%%%%%%%%%%%%%
%%% HERE YOU CAN ADD YOUR OWN MACROS AND ENVIRONMENTS IN THE PREAMBLE %%%%%
%%%%%%%%%%%%%%%%%%%%%%%%%%%%%%%%%%%%%%%%%%%%%%%%%%%%%%%%%%%%%%%%%%%%%%%%%%%

% Add your macros here.


\begin{document}

%%%%%%%%%%%%%%%%%%%%%%%%%%%%%%%%%%%%%%%%%%%%%%%%%%%%%%%%%%%%%%%%%%%%%%%%%%%
%%%%%%%%%%%% THE FOLLOWING GENERATES THE HEADER %%%%%%%%%%%%%%%%%%%%%%%%%%%
%%%%%%%%%%%% DO NOT TOUCH THIS %%%%%%%%%%%%%%%%%%%%%%%%%%%%%%%%%%%%%%%%%%%%
%%%%%%%%%%%%%%%%%%%%%%%%%%%%%%%%%%%%%%%%%%%%%%%%%%%%%%%%%%%%%%%%%%%%%%%%%%%

\thispagestyle{firststyle}

\noindent
\hspace{0.3cm}{\huge\textbf{\coursename}}

\noindent
\rule{\textwidth}{1pt}

\vspace{0.3cm}

\noindent
\begin{tabularx}{\textwidth}{lXl}
  \multirow{3}{*}{\textbf{\huge\homework}} && {\Large\textbf{\studentname}} \\
&&\\[-0.35cm]
  && {\Large\texttt{\email}} \\
&&\\[-0.2cm]
\cline{3-3}
&&\\[-0.2cm]
\end{tabularx}

\vspace{0.2cm}
\noindent
\rule{\textwidth}{1pt}

\vspace{0.5cm}

\pagestyle{fancy}

\section{What the program does}
The program calculates the factorial of any integer $n$, that is the product $\Pi_{i=1}^{n}i$.

\section{How to compile}
You need GNU Multi Precision Library (GMP) to compile the program. In Ubuntu, run \verb!sudo apt-get install libgmp-dev! to install it.

\begin{mdframed}[style=MyFrame]
Compile with \verb!gcc fact.c -std=c99 -pthread -lgmp -O3 -o fact!
\end{mdframed}

\section{How to use}

The program requires two flags \verb!-i!, the number of integers processed by each thread and \verb!-t!, the number of threads. For example, to calculate
$300 = 30 * 10$ factorial, type something like \verb!./fact -i 30 -t 10!.

\begin{mdframed}[style=MyFrame]
Type \verb!./fact! without any arguments to get help.
\end{mdframed}

\section{Speed of execution}
\begin{verbatim}
time ./fact -i 1000 -t 100
Computing 100000!
(Lots of output)
real   0m10.041s
user   0m4.375s
sys    0m0.072s
\end{verbatim}

If you use the flag \verb!-d!, you get the execution time of each thread. It might for example look like this:

\begin{verbatim}
> ./fact -i 100000 -t 4 -d
Computing 400000!
Thread 0 finished in 3 ms.
Thread 2 finished in 3 ms.
Thread 1 finished in 3 ms.
Thread 3 finished in 4 ms.
(Lots of output)
\end{verbatim}

\section{Limitations}
The program limits the number of threads that can be created. The default limit is 300 threads, you get change it by redefining \verb!MAX_THREADS!. The operating system also has an internal limit for the number of threads per process. This limit can be changed by editing the file \verb!/proc/sys/kernel/threads-max!.

\end{document}